%%%%%%%%%%%%%%%%%%%%%%%%%%%%%%%%%%%%%%%%%
% University/School Laboratory Report
% LaTeX Template
% Version 3.0 (4/2/13)
%
% This template has been downloaded from:
% http://www.LaTeXTemplates.com
%
% Original author:
% Linux and Unix Users Group at Virginia Tech Wiki 
% (https://vtluug.org/wiki/Example_LaTeX_chem_lab_report)
%
% License:
% CC BY-NC-SA 3.0 (http://creativecommons.org/licenses/by-nc-sa/3.0/)
%
%%%%%%%%%%%%%%%%%%%%%%%%%%%%%%%%%%%%%%%%%

%----------------------------------------------------------------------------------------
%	PACKAGES AND DOCUMENT CONFIGURATIONS
%----------------------------------------------------------------------------------------

\documentclass{article}

%\usepackage{mhchem} % Package for chemical equation typesetting
%\usepackage{siunitx} % Provides the \SI{}{} command for typesetting SI units

\usepackage{graphicx} % Required for the inclusion of images
\usepackage[top=1in,bottom=1in,right=1in,left=1in]{geometry}% Set the margins

%Multiple column picture packages
\usepackage{caption}
\usepackage{subcaption}

%Add code formating
\usepackage{listings}
\lstset{tabsize=2}

%Add support for floating images
\usepackage{float}

%Define the style for VHDL
\lstdefinelanguage{VHDL}{
  morekeywords={
    library,use,all,entity,is,port,in,out,end,architecture,of,
    begin,and
  },
  morecomment=[l]--
}

%Give the VHDL code color
\usepackage{xcolor}
\colorlet{keyword}{blue!100!black!80}
\colorlet{comment}{green!90!black!90}
\lstdefinestyle{vhdl}{
  language     = VHDL,
  basicstyle   = \ttfamily,
  keywordstyle = \color{keyword}\bfseries,
  commentstyle = \color{comment}
}

\usepackage{amssymb}
\usepackage{amsmath}

%Highlight command
\usepackage{tikz}
\usepackage{xspace}
\usetikzlibrary{decorations.pathmorphing}
\newcommand\hl[1]{%
    \tikz[baseline,%
      decoration={random steps,amplitude=1pt,segment length=15pt},%
      outer sep=-15pt, inner sep = 0pt%
    ]%
   \node[decorate,rectangle,fill=yellow,anchor=text]{#1\xspace};%
}%

% Create the header and footer
\usepackage{fancyhdr}
\pagestyle{fancy}
\lhead{CprE 583}
\rhead{Blake Vermeer, Kris Hall, Rohit Zambre}
\renewcommand{\footrulewidth}{0.4pt}

\setlength\parindent{0pt} % Removes all indentation from paragraphs

\renewcommand{\labelenumi}{\alph{enumi}.} % Make numbering in the enumerate environment by letter rather than number (e.g. section 6)

%\usepackage{times} % Uncomment to use the Times New Roman font

%----------------------------------------------------------------------------------------
%	DOCUMENT INFORMATION
%----------------------------------------------------------------------------------------

\title{MP-2 Write-Up} % Add title here...

\author{Blake \textsc{Vermeer}\\
		Kris \textsc{Hall}\\
		Rohit \textsc{Zambre}} % Author name

\date{\today} % Date for the report

\begin{document}

\maketitle % Insert the title, author and date

\begin{center}
\begin{tabular}{l r}
Date Due: & October 10, 2014 \\ % Date the assignment is due
%Partners: & James Smith \\ % Partner names (optional)
Instructors: & Joseph Zambreno % Instructor/supervisor
\end{tabular}
\end{center}

% If you wish to include an abstract, uncomment the lines below
% \begin{abstract}
% Abstract text
% \end{abstract}


% If you need to include a figure, copy the lines below
%\begin{figure}[H]
%	\begin{center}
%		\includegraphics[scale=0.35]{ADD FILE LOCATION OF PICTURE FILE HERE}
%		\caption{ADD CAPTION HERE}
%	\end{center}
%\end{figure}


%If you need to include VHDL code, copy the lines below and fill in the VHDL code
%\begin{center}
%	\begin{lstlisting}[style=vhdl]
%		VHDL CODE GOES HERE.	
%	\end{lstlisting}
%\end{center}

%----------------------------------------------------------------------------------------
%	SECTION 1
%----------------------------------------------------------------------------------------

\section{Objective}



\section{Platform Overview}



\section{Basic Scanning}



\section{String Detection}


\section{String Counting}



\section{Message Return}



\section{Bonus - Inner-Packet Counting}








% If you have more than one objective, uncomment the below:
%\begin{description}
%\item[First Objective] \hfill \\
%Objective 1 text
%\item[Second Objective] \hfill \\
%Objective 2 text
%\end{description}

%\subsection{Definitions}
%\label{definitions}
%\begin{description}
%\item[Stoichiometry]
%The relationship between the relative quantities of substances taking part in a reaction or forming a compound, typically a ratio of whole integers.
%\item[Atomic mass]
%The mass of an atom of a chemical element expressed in atomic mass units. It is approximately equivalent to the number of protons and neutrons in the atom (the mass number) or to the average number allowing for the relative abundances of different isotopes. 
%\end{description} 
 
%----------------------------------------------------------------------------------------
%	SECTION 2
%----------------------------------------------------------------------------------------




%----------------------------------------------------------------------------------------
%	SECTION 4
%----------------------------------------------------------------------------------------
\section{Conclusion}



%----------------------------------------------------------------------------------------
%	BIBLIOGRAPHY
%----------------------------------------------------------------------------------------
%\newpage

%\bibliographystyle{unsrt}

%\bibliography{sample}

%----------------------------------------------------------------------------------------


\end{document}